\documentclass[11pt, a4paper]{MATH2421}
\usepackage{verbatim}
\usepackage{fancyhdr}
\usepackage{booktabs}
\usepackage{setspace}
\usepackage{amsmath,mathrsfs}
\usepackage{multicol}
\usepackage{amssymb}
\usepackage{graphicx}
\usepackage{caption}
\usepackage{subcaption}
\usepackage{array}
\usepackage{xcolor}
\usepackage{float}
\usepackage{enumitem}
\usepackage{mathcomp}
\usepackage{tabularx}
\usepackage{wasysym}
\usepackage{pbox}
\usepackage{tikz}
\usepackage{mathtools}
\usetikzlibrary{matrix}
\usepackage[normalem]{ulem}
\usepackage{multirow}
\usepackage[linesnumbered, ruled, boxed]{algorithm2e}
\SetKwRepeat{Do}{do}{while}

\title{Chapter 1}
\subtitle{Combinatorial Analysis}

\newtheorem*{theorem}{Theorem}
\newcommand{\eg}{{\bf {Example: }}}
\newcommand{\sol}{{\bf {Solution: }}}
\renewcommand\qedsymbol{$\blacksquare$}

\begin{document}
\begin{spacing}{1.3}

    \section{Principle of Counting}

    \begin{theorem}
        (basic principle of counting) two experiments: 
        one has $m$ outcomes, the other has $n$ outcomes, 
        then together there are $mn$ outcomes of the two 
        experiments.
    \end{theorem}

    \begin{proof}
        prove by enumerating all $(i,j)$ pairs of 
        possible outcomes.
    \end{proof}

    \begin{theorem}
        (generalized BPOC) $r$ experiments: 
        each has $n_1,n_2\cdots n_r$ outcomes, in total 
        $n_1\cdot n_2\cdots n_r$ outcomes.
    \end{theorem}

    \eg How many different 7-place license plates are possible, 
        if first 3 are letters, final 4 are numbers, don't allow 
        repetition?

    \sol $26\times 25\times 24\times 10\times 9\times 8
    \times 7=78,624,000$

    \section{Permutation}

    \begin{theorem}
        Permutation: different ordered arrangements.
        $n$ distinct objects, total number of different 
        arrangement is $n!$, with the convention $0!=1$.
    \end{theorem}

    \begin{theorem}
        $n$ objects, of which $n_1$ are alike, $n_2$ are alike, 
        $\cdots$, $n_r$ are alike, there are
        $$\frac{n!}{n_1!n_2!\cdots n_r!}$$ 
        different permutations.
    \end{theorem}

    \eg Ways to rearrange ``Mississippi''?

    \sol $\dfrac{11!}{1!4!4!2!}=34,650$.

    \begin{theorem}
        For $n$ people sitting in a circle: there are 
        $$\frac{n!}{n}=(n-1)!$$
        different arrangements.
    \end{theorem}

    \eg How many different ways can $n$ different pearls 
    string in a necklace?

    \sol $\dfrac{(n-1)!}{2}$, since the necklace can be 
    flipped.(mirrored)

    \section{Combinations}

    \begin{theorem}
        $n$ distinct objects, choose $r$ to form a group:
        $$_nC_r={n\choose r}=\frac{n!}{r!(n-r)!}$$
    \end{theorem}

    {\bf Some properties:}
    $${n\choose r}={n\choose n-r},\ 
    {n\choose 0}={n\choose n}=1$$
    And by convention, when $n\ge 0$ and 
    $r<0$ or $r>n$, ${n\choose r}=0$.

    \eg From a party of 3 from 20 people?

    \sol ${20\choose 3}=1140$

    \eg What if $A$ and $B$ can't be chosen together?

    \sol (1) minus the situation that they are both chosen: 
    ${20\choose 3}-{18\choose 1}{2\choose 2}=1122$,
    or (2) neither of them are chosen plus one of them chosen: 
    ${18\choose 3}+{18\choose 2}{2\choose 1}=1122$.

    \eg $m$ antennas are defective and $n-m$ are functional.
    The system works if no two defectives are consecutive.
    How many different linear orderings?

    \sol Consider inserting $m$ defective ones into 
    the space of $n-m$ functional ones. 
    $n-m$ antennas provide $n-m+1$ spaces, 
    so we just choose $m$ of them: ${n-m+1\choose m}$.

    \begin{theorem}
        For $1\le r\le n$, 
        $${n\choose r}={n-1\choose r-1}+{n-1\choose r}$$.
    \end{theorem}

    \begin{proof}
        (Algebra proof) 
        $$RHS=\frac{(n-1)!}{(r-1)!(n-r)!}+
        \frac{(n-1)!}{r!(n-r-1)!}
        =\frac{(n-1)!{\red{r}}}{(r-1)!(n-r)!{\red{r}}}+
        \frac{(n-1)!{\red{(n-r)}}}{r!(n-r-1)!{\red{(n-r)}}}
        =\frac{(n-1)!\cdot \red{[r+(n-r)]}}{r!(n-r)!}=LHS$$

        (Combinatorial proof)

        LHS: number of choosing $r$ balls from $n$ balls. 
        ${n\choose r}$

        RHS: consider whether to choose the 1st ball:
        \begin{itemize}
            \item choose 1st ball, then choose $r-1$ from remaining
            $n-1$ balls
            \item don't choose 1st ball, then choose $r$ from remaining
            $n-1$ balls
        \end{itemize}
        combine the two situations, $\disp {n-1\choose r-1}+{n-1\choose r}$.

    \end{proof}

    \begin{theorem}
        (Binomial Theorem) 
        $n$ be nonnegative integer, then
        $$(x+y)^n=\sum_{k=0}^{n}{n\choose k}x^ky^{n-k}$$
    \end{theorem}

    \begin{proof}
        (Combinatorial Proof) for $x^ky^{n-k}$, it means 
        there are $k$ brackets contribute $x$, while 
        remaining $n-k$ brackets contribute $y$.
    \end{proof}

    \eg How many subsets does a set of size $n$ have?

    \sol There are ${n\choose k}$ subsets of size $k$,
    so add them together:
    $$\sum_{k=0}^{n}{n\choose k}=(1+1)^n=2^n.$$

    \eg Proof: $\disp \sum_{k=0}^{n}(-1)^k{n\choose k}=0.$

    \sol Let $x=-1, y=1,$ the problem $=[(-1)+1]^n=0$.
    

    \section{Multinomial Coefficients}
\end{spacing}
\end{document}
