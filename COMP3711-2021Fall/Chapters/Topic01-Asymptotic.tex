\documentclass[11pt, a4paper]{COMP3711}
\usepackage{verbatim}
\usepackage{fancyhdr}
\usepackage{booktabs}
\usepackage{setspace}
\usepackage{amsmath,mathrsfs}
\usepackage{multicol}
\usepackage{amssymb}
\usepackage{graphicx}
\usepackage{caption}
\usepackage{subcaption}
\usepackage{array}
\usepackage{xcolor}
\usepackage{float}
\usepackage{enumitem}
\usepackage{mathcomp}
\usepackage{tabularx}
\usepackage{wasysym}
\usepackage{pbox}
\usepackage{tikz}
\usepackage{mathtools}
\usetikzlibrary{matrix}
\usepackage[normalem]{ulem}
\usepackage{multirow}
\usepackage[linesnumbered, ruled, boxed]{algorithm2e}
\SetKwRepeat{Do}{do}{while}

\title{Topic 1}
\subtitle{Asymptotic}

\begin{document}
\begin{spacing}{1.5}
    
    \section{Algorithm}

    An \textbf{algorithm} is an explicit, precise, unambiguous, mechanically-executable
    sequence of elementary instructions. To evaluate an algorithm, we measure:
    \begin{itemize}
        \item {\bf Memory (Space Complexity)}: 
        all space used except for holding inputs
        \item {\bf Running time (Time Complexity)}  
    \end{itemize}

    In this course, we measure algorithms {\it analytically}, i.e., 
    depends only on the algorithms, without considering actual
    implementations, hardwares, etc. 

    However, it is difficult and rarely that we can say
    ``one algo is better than the other'', since that usually 
    depends on input size, input data(even for same size), etc.

    \section{Time Complexity}

    Usually, we measure running time(time complexity) as the num
    of machine instructions, such as addition, multiplication,
    swap(as used in sorting analysis)... We describe running time
    as a function of input size: $T(n)$.

    There are three commonly-used asymptotic notations:
    \begin{itemize}
        \item {\bf Upper bounds:} $T(n)=O(f(n))$, if
        $\exists c>0, n_0\ge 0$ such that 
        $\forall n\ge n_0, T(n)\le c\cdot f(n)$.
        \item {\bf Lower bounds:} $T(n)=\Omega(f(n))$, 
        if $f(n)=O(T(n))$.
        \item {\bf Tight bounds:} $T(n)=\Theta(f(n))$,
        if both $T(n)=O(f(n))$ and $T(n)=\Omega (f(n))$.
    \end{itemize}

    Here are some notes for above notations: 
    First, more accurate expression should be $T(n)\in O(f(n))$,
    but we often use $=$ for simplicity, which means ``is'', not ``equal''.
    Second, these notations is not properly definable using limits.
    One may think that $f(n)=O(g(n))$ is equivalent to 
    $\disp \lim_{n\rar \infty} \frac{f(n)}{g(n)}<\infty$, 
    but a counterexample can easily be found like 
    $f(n)=(2+(-1)^{n})g(n)$, in which case the limit does not exist.

    I will omit examples here, but I'd like to list some interesting facts.
    (1) $2^{10n}$ is not $O(2^n)$, since it is $(2^n)^{10}$. 
    (2) $\Theta (f(n)+g(n))=\Theta(\max(f(n), g(n)))$.
    (3) $\disp \sum_{i=1}^{n}\frac{1}{i}=O(\log n)$, which is 
    called {\it Harmonic Series}.
    (4) $\log(n!)=\Theta (n\log n)$.

    For a certain algorithm, different inputs can cause different
    performances, even with same input size.
    For insertion sort, input an already sorted list requires
    no additional swaps, which gives $\Theta(n)$, and this is 
    called {\bf best case}; input an inversely sorted list
    gives $T(n)=\sum_{i=2}^{n}(i-1)=\Theta(n^2)$, this is 
    {\bf worst case}; if we average over all possible inputs
    for a certain size $n$, assuming same probability distribution
    on these inputs, then the result running time is called 
    {\bf average case}.

    Generally, average case analysis is rather complicated.
    In insertion sort, we assume each of the $n!$ permutations
    is distributed equally likely. With some probability 
    knowledge we will know it's $\Theta(n^2)$. I will 
    give brief proof in last page of this note if you are interested.

    Let's have a summary of three kinds of analysis: 
    (1) best case is ideal so that it is useless;
    (2) average case is sometimes used but requires
    complicated analysis; 
    (3) {\bf worst case} is commonly used, since it gives 
    running time guarantee {\bf independent of actual input}.
    In this course, {\bf Worst-case analysis is the default},
    but it is not perfect: some algorithms with bad worst-case 
    running time actually work very well in practice, 
    since worst case input rarely occurs.

    When we say an algorithm’s worst case running time is $O(f(n))$,
    we mean {\bf on all inputs of size $n$}, the algorithm’s 
    running time is $O(f(n))$, but there is no need to really 
    {\it find} the worst input to prove.

    When we say an algorithm’s worst case running time is $\Omega(f(n))$,
    we mean {\bf there exists at least one input of size $n$}, the algorithm’s 
    running time is $\ge c\cdot f(n)$. We mainly use this to prove 
    the big-Oh analysis is tight.

    To understand above two paragraphs, again consider insertion
    sort: it runs in $\le \frac{n(n-1)}{2}$ time for all inputs 
    of size $n$, so it is $O(n^2)$, it {\bf requires} 
    $\frac{n(n-1)}{2}$ time if items are reversed, 
    so it is $\Omega(n^2)$. To combine, it runs in $\Theta(n^2)$ time.

    {\it This is the end of class note. Last modified: Sep 9\\
    Sep 9: Fixed some typos.}


    \newpage
    {\it Brief proof for average case time complexity of 
    insertion sort:} \\
    Firstly, one can show that the number of ``swaps'' 
    is equals to the number of {\bf inversions}.(proof by induction in 
    lecture slide divide \& conquer)\\
    So now we know the running time for a certain input
    will be $\Theta(n+I)$, 
    where $I$ is the number of inversions of the original array.\\
    Here, we define $X_{ij}$ to be 1 if $a[i]$ and $a[j]$ form 
    an inversion and 0 otherwise. So an given input of size $n$
    will have $n(n-1)/2$ different $X_{ij}$s.\\
    Now, we can express $I$ as: $I=\sum X_{ij}$. But 
    remember we are interested in the {\bf expected number 
    of inversions} in the array, since we're looking for average 
    running time of all inputs. This is also simple by 
    linearity of expectation: $E(I)=E(\sum X_{ij})=\sum E(X_{ij})$.\\
    That's a good one, $E(X_{ij})$ is the expected value
    of $X_{ij}$, of course it is $1\cdot P(X_{ij}=1)=0.5$, 
    since we have assumed $n!$ permutations are equally likely.\\
    Thus, $E(I)=\sum (1/2)$, and there are $n(n-1)/2$ terms,
    which gives $E(I)=n(n-1)/4=\Theta(n^2)$.\\
    To sum up, on expectation the runtime will be
    $\Theta(n^2+n)=\Theta(n^2)$, This explains why the average-case 
    behavior of insertion sort is $\Theta(n^2)$.
\end{spacing}
\end{document}
