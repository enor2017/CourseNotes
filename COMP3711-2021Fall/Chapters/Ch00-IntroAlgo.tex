\begin{spacing}{1.3}
    \section{What is an Algorithm?}
    \begin{definition}
        An {\bf algorithm} is a recipe for doing something.
        An {\bf algorithm} is an explicit, precise, unambiguous, mechanically-executable 
        sequence of elementary instructions.
    \end{definition}

    Let's look at the algorithm that adding two numbers, which you use every day.

    {\bf Input:} Two numbers $x$ and $y$, each consisting of $n$ digits: 
    $x=\overline{x_nx_{n-1}\cdots x_1}, y=\overline{y_ny_{n-1}\cdots y_1}$.
    They maybe very long.

    {\bf Output}: A number $z=\overline{z_{n+1}z_n\cdots z_1}$ such that $z=x+y$.

    The pseudocode is given below:
    \begin{algorithm*}[htbp]
        \caption{Add-Two-Numbers($x, y$)}
        $c\lar 0$ \qquad \tcp{$c$ is carry-in}

        \For{$i\lar 1$ to $n$}{
            $z_i\lar x_i+y_i+c$

            \eIf{$z_i\ge 10$}{
                $c\lar 1,\ z_i\lar z_i-10$
            }{
                $c\lar 0$
            }
        }
        $z_{n+1}\lar c$
    \end{algorithm*}


    \vspace{0.3in}
    \section{Insertion Sort}

    {\bf Problem:} Given an array $A[1\cdots n]$ of elements, sort 
    the array in ascending order.

    Well, to design an algorithm, what I typically do is to think how we will 
    solve this problem by hand. Consider when you play poker, you already 
    have a hand of sorted cards, and when you get a new card, you {\it choose 
    a proper position to insert it}. For example, you currently have cards:
    $$[1,3,4,5,9]$$
    and then you get a new card 7, you scan over the cards, and find out 
    that you should insert 7 between 5 and 9. This is basically how we 
    do in insertion sort algorithm.

    Let's do a more completed demo. At the beginning, you have an array
    $$[5,1,8,3]$$
    At step 1, you get 5, since you have no number before, 5 itself is already sorted,
    so now in your hand, you have $$[5]$$

    At step 2, you get 1, and you find out 1 should be inserted before 5, since $1<5$.
    Now in your hand, you have $$[1,5]$$

    At step 3, you get 8, and you find out 8 should be inserted after 5, so now in your hand,
    $$[1, 5, 8]$$

    At step 4, you get 3, it should be inserted between 1 and 5, so now in your hand, 
    $$[1,3,5,8]$$

    This is not difficult right? However, to translate it into a formal algorithm, 
    we need to figure out two things:
    \begin{itemize}
        \item How should we find the position to insert a number, and
        \item How should we actually ``insert'' the number into the array?
    \end{itemize}

    For the first point, when we get a new number $x$, we can scan the numbers we already had, 
    from right to left, until we meet a number $y$ such that $y\le x$, then 
    we know we should insert $x$ after $y$.

    For the second point, since this is an array, the only thing we can do is to 
    ``move all items one position afterwards'', and then put the new number into the vacant position.
    For example, insert $3$ into $[1,5,8]$, we first move 5 and 8 backwards, say 
    $$[1, \_, 5, 8]$$
    then, put 3 inside, $$[1,3,5,8]$$
    Alternatively, we can continuously swap from right to left, until $y\le x$ is met. 
    Notice that this method is preferred, since we combine the process of ``finding 
    insertion position'' and ``insert number into array'' into one!
    As an example, insert 3 into $[1,5,8]$, we first append 3 at last,
    $$[1,5,8,3]$$
    since $8>3$, we swap 8 with 3, $$[1,5,3,8]$$,
    since $5>3$, we swap 5 with 3, $$[1,3,5,8]$$,
    since $1<3$, now 3 is at the correct position, no further swap is required.

    Up to now, you should have known the basic idea as well as some different implementations 
    of Insertion Sort algorithm. To describe it using the last method with pseudocode,
    
    \begin{algorithm*}[htbp]
        \caption{Insertion-Sort($A$)}

        \tcp{don't need to sort first item, so we proceed with second item, here $i$ points to 
        the new item we get}

        \For{$i\lar 2$ to $n$}{
            $j\lar i-1$ \qquad \tcp{$j$ points to right most item}

            \tcp{As long as $A[j]>A[j+1]$, we continuously swap $j$ and $j+1$}
            \While{$j>1$ and $A[j]>A[j+1]$}{
                swap $A[j]$ and $A[j+1]$

                $j\lar j-1$\qquad \tcp{continue to move left, $j$ decreases}
            }
        }
    \end{algorithm*}

    After designing an algorithm, we need to prove its correctness, which, in this example,
    is to prove this algorithm can actually sort items in ascending order.
    The correctness of this algorithm is hopefully, obvious, and you may find a 
    very brief proof on lecture slide, you can also prove it by induction.

    Apart from correctness, we also need to examine its performance.
    Usually, we care about its {\bf time complexity} and {\bf space complexity},
    you can think of them as, respectively, how long will the algorithm take 
    for a certain amount of input, and how much space/memory will the algorithm occupy
    during execution. We will formally introduce them in next topic.

    For now, (and actually most time in this course), we only examine the running time, i.e., 
    {\bf time complexity} of Insertion Sort. Hopefully, it is not surprising that 
    {\it line 3} above is the dominant part of running time, since it resides in the 
    most inner part of the two loops. Thus, we can simply calculate how many times 
    will line 3 be executed, and use it to serve as the running time of this algorithm.

    Let's have a look at line 3, the ``while'' loop will repeat as long as 
    $j\ge 1$ and $A[j]>A[j+1]$, i.e., as long as the current number is smaller than 
    the left number, we will swap them, and continue to check next one.
    So the extreme case is if the current number, $A[j]$, is the {\it smallest} among 
    all previous numbers, $A[1\cdots j-1]$, then we will perform $j-1$ swaps.
    Therefore, this line will run at most
    $$\sum_{i=2}^{n}(i-1)=\frac{n(n-1)}{2}$$
    times. This immediately follows that {\bf in the worst case, the running time of 
    Insertion Sort is $\dfrac{n(n-1)}{2}$.}

    However, that's the worst case, how about other cases? Let's now assume the opposite 
    to what we assumed just now, that is, if the current number, $A[j]$, 
    is the {\it largest} among all previous numbers, $A[1\cdots j-1]$, then we will not 
    perform any swaps, so line 3 only run $(n-1)$ times.(every time we do a checking 
    and quit the while loop immediately)

    From above analysis, we can observe that in different cases, i.e., when the data vary,
    the running time of an algorithm also varies. We will come back to this point 
    in next topic. One thing worths mentioning is that in our worst case above, 
    we assume the current item is the smallest one, so the input data is 
    ``{\it reversely sorted}''; will in our best case above, we assume the current one is 
    largest, so the input data is {\it already sorted}.
    




    \vspace{0.3in}
    \section{About pseudocode}

    In this course, we use pseudocode to explicitly describe algorithms.
    There are lots of reasons for this. On one hand, real codes, such as 
    Java, C++ programs, are too hard to read and implement. For example,
    ``sort $n$ intervals according to their start time'' is not easy to 
    implement in most programming languages, but in pseudocode, we can 
    just write like this without any ambiguity. On the other side, 
    if we use natural languages, they are, usually, not descriptive enough,
    or they may be quite ambiguous sometimes.

    Pseudocode codes have different styles, and in this course, we normally 
    follow the below rules:
    \begin{itemize}
        \item {\bf Use standard keywords}: if/then/else, for, while, return, repeat/until 
        \item {\bf Use standard notations}: 
            \begin{itemize}
                \item Use ``$variable \lar value$'' to assign value to a variable,
                \item Use ``$variable = value$'' as an ``assertion'', which is a true-of-false 
                statement to check if they are equal
                \item Use ``$Array[index]$'' to represent arrays, 
                \item Use ``$function(arguments)$'' to represents functions, $\cdots$
            \end{itemize}
        \item {\bf Make clear indentations}: If you know Python, you know how to do it
        \item {\bf Use mathematical notations}:
            \begin{itemize}
                \item Use ``$i\lar i+1$'' instead of $i=i+1$, or $i++$
                \item Use ``$x\cdot y, x\ {\rm mod}\ y$'' instead of $x*y, x\% y$
                \item Use ``$\sqrt{x}, a^b$'' instead of sqrt($x$), power($a,b$)
                \item Use ``$\lfloor n/2 \rfloor$'' instead of $n/2$: we don't have implicit truncation
                \item Use ``remove first item from list'' instead of List.removeFirst()
            \end{itemize}
        \item {\bf Hide unnecessary data structures/sub-routines, make them becomes ``black-box''}:
            \begin{itemize}
                \item Directly say ``use Insertion Sort to sort array $A[1\cdots n]$'', instead of 
                re-write whole procedure of Insertion Sort.
                \item Directly say ``create a heap'', instead of re-describe how to construct 
                and implement a heap
                \item Directly say ``$x=$ the maximum element in $A$'' or 
                $\disp M=\max_{1\le i\le n} A[i]$ instead of writing a loop to find.
                (But notice this takes $O(n)$ time, not $O(1)$)
            \end{itemize}
    \end{itemize}

    In order to write pseudocode for this course, I found it handy to type it in \LaTeX,
    as an example, the code below 

    \begin{minted}[linenos, autogobble]{tex}
        \begin{algorithm*}[htbp]
            \caption{Example-Pseudocode($A[1\cdots n]$)}

            \tcp{This is a comment.}

            $result \leftarrow 0$

            \For{$i\leftarrow 1$ to $n$}{
                \eIf{$i=1$}{
                    $result \leftarrow result + 1$
                }{
                    $result \leftarrow result + i$
                }
                \If{$result > 1000$}{
                    break
                }
            }
            return $result$
        \end{algorithm*}
    \end{minted}

    will be rendered as: 
    \begin{algorithm*}[htbp]
        \caption{Example-Pseudocode($A[1\cdots n]$)}

        \tcp{This is a comment.}

        $result \leftarrow 0$

        \For{$i\leftarrow 1$ to $n$}{
            \eIf{$i=1$}{
                $result \leftarrow result + 1$
            }{
                $result \leftarrow result + i$
            }
            \If{$result > 1000$}{
                break
            }
        }
        return $result$
    \end{algorithm*}
    
    I believe the syntax is relatively easy to understand.

\end{spacing}